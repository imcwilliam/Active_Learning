\documentclass[a4paper,11pt]{article}
\usepackage{amsmath,amsfonts,amsthm,bm,listings,graphicx,gensymb,float,subcaption,multicol,wrapfig,stfloats,breqn}
\usepackage[export]{adjustbox}
\usepackage[labelfont=bf]{caption}
\usepackage[margin=0.75in]{geometry}
\graphicspath{ {/Users/ianmcwilliam/Desktop/MSc/Active_Learning/} }
\usepackage{titling}
\setlength{\droptitle}{0cm}

\begin{document}
\title{Informatics Project Proposal - Deep Learning for Active Models}
\author{Ian McWilliam s0904776}
\date{}
\maketitle

\section{Introduction}
\textit{Active learning} refers to the learning paradigm wherein machine learning algorithms actively select or `query' the samples from which it learns; in the case of deep learning this constrasts with the default approach of randomly sampling `batches' of labelled training samples. The motivation for active learning is that, by allowing it to select the samples from which it learns, the algorithm can achieve superior generalization performance from a smaller number of training samples than if the samples had been chosen randomly. In domains where unlabelled data is abundant but obtaining labels is expensive, active learning can be used to reduce the cost associated with training a deep model, as the active approach allows the deisgner to obtain labels only for the samples which will be most beneficial to learning. 

There are a variety of methods by which the algorithm can query datapoints, however they generally focus on finding the points in the input space that the algorithm is most `uncertain' about, allowing the algorithm to fill what could be seen as gaps in its knowledge of the domain. A related field is that of \textit{curriculum learning}, which explores how the learning process can be improved by presenting training samples to the algorithm in a meaningful order (with the order defining a `curriculum'). Motivated by the way in which humans and animals learn, the learning curriculum generally begins with `easy' examples, with the difficulty of the examples increasing as training progresses; Bengio et al show that learning with a curriculum can improve overall performance on a variety of tasks etc. It is interesting to note that, while similar, active and curriculum learning are somewhat opposed in their learning philopsophies, with the former focussing on learning from uncertain/difficult samples and the latter focussing beginning with easy samples before continuing to more difficult ones. 

 note that this is in some ways diametrically opposit

\section{Purpose}

\section{Background}

\section{Methods}

\section{Evaluation}

\section{Outputs}

\section{Workplan}






\begin{thebibliography}{1}
\bibitem{O'Keefe 71}
J.O'Keefe, J.Dostrovsky, ``The hippocampus as a spatial map. Preliminary evidence from unit activity in the freely-moving rat'', \textit{Brain Research}, Issue 34, pp 171-175, 1971
\end{thebibliography}


\end{document}